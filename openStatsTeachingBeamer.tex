%	Use the standard preamble for Beamer slides of all 
%		statsTeachR modules
%		(\input, not \include, as \include can't access 
%		things in higher-level directories since it needs 
%		write permission there, which it doesn't have; and 
%		in some settings the preamble may be in a higher-level
%		directory than the source file.)
%	This path assumes the preamble is in the parent directory,
%		modify this if that changes.
\input{standard_beamer_preamble}
\usepackage{multicol}

%	The following variables are assumed by the standard preamble:
%	Global variable containing module name:
\title{Open Resources for Teaching Statistics }
%	Global variable containing module shortname:
%		(Currently unused, may be used in future.)
\newcommand{\ModuleShortname}{}
%	Global variable containing author name:
\author{Andrew Bray, Nicholas G Reich}
%	Global variable containing text of license terms:
\newcommand{\LicenseText}{Made available under the Creative Commons Attribution-ShareAlike 3.0 Unported License: http://creativecommons.org/licenses/by-sa/3.0/deed.en\textunderscore US }
%	Instructor: optional, can leave blank.
%		Recommended format: {Instructor: Jane Doe}
\newcommand{\Instructor}{UMass-Amherst Probability and Statistics Seminar Series}
%	Course: optional, can leave blank.
%		Recommended format: {Course: Biostatistics 101}
\newcommand{\Course}{7 April 2014}



%	******	Document body begins here	**********************

\begin{document}

%	Title page
\begin{frame}[plain]
	\titlepage
\end{frame}

%	******	Everything through the above line must be placed at
%		the top of any TeX file using the statsTeachR standard
%		beamer preamble. 

%%%%%%%%%%%%%%%%%%%%%%%%%%%%%%%%

\begin{frame}{Many degrees of Openness}

~ reproducible research important tie to open access/transparency
~ statisticians are the arbiters of science
~ other fields ahead of stats in open-access publishing
~ stats may be ahead in terms of publishing open-access


stats textbooks
~ OpenIntro
~ Lavine
~ Collaborative Statistics
~ Introductory Statistics (http://www.saylor.org/site/textbooks/Introductory\%20Statistics.pdf)

\end{frame}

%%%%%%%%%%%%%%%%%%%%%%%%%%%%%%%%

\begin{frame}{Many degrees of Openness}

\begin{multicols}{2}
\includegraphics[height = 1in]{OAlogo.png}

\columnbreak

\begin{block}{Common characteristics}
\begin{itemize}
	\item Online
\pause
	\item No access restrictions
\pause
	\item No cost
\pause
	\item Free-to-modify
\end{itemize}
\end{block}
\end{multicols}

\end{frame}

%%%%%%%%%%%%%%%%%%%%%%%%%%%%%%%%

\begin{frame}{Many degrees of Openness}

\includegraphics[height = 1.5in]{FOAS.png}

\begin{itemize}
	\item Free software
\pause
	\item Open access publishing
\pause
	\item Open source educational resources
\end{itemize}

\end{frame}

%%%%%%%%%%%%%%%%%%%%%%%%%%%%%%%%


\begin{frame}{OpenIntro}

About 10 minutes of content

\end{frame}

%%%%%%%%%%%%%%%%%%%%%%%%%%%%%%%%


\begin{frame}{What is statsTeachR.org?}

\begin{block}{The basics}
\begin{itemize}
        \item a new, open-access, online repository with modular lesson plans
        \item materials targeted at undergraduates and graduates 
        \item topics: stats, biostats, statistical computing with R
\end{itemize}
\end{block}

\begin{block}{Modules}
\begin{itemize}
        \item each curricular ``module'' focuses on teaching a particular statistical subject or concept
        \item can be browsed \`a la carte
\end{itemize}
\end{block}

\end{frame}

%%%%%%%%%%%%%%%%%%%%%%%%%%%%%%%%

\begin{frame}{statsTeachR.org}

Website tour: \href{http://statsTeachR.org}{statsTeachR.org}

\end{frame}

%%%%%%%%%%%%%%%%%%%%%%%%%%%%%%%%

\begin{frame}{statsTeachR enables collaborative curriculum development}

\begin{block}{Some ongoing experiments for building curriculum}
\begin{itemize}
        \item graduate students are creating modules as final projects in core Biostat classes at UMass
        \item co-development of materials for similar classes (UMass-Amherst and Columbia Univ) 
\end{itemize}
\end{block}


\end{frame}

%%%%%%%%%%%%%%%%%%%%%%%%%%%%%%%%

\begin{frame}{Discussion questions}

\begin{itemize}
        \item where does the time/money/person-hours come from to generate these resources?
        \item how to incentivize these kinds of contributions appropriately?
        \item how to control quality of resources?
\end{itemize}
\end{frame}

%			\item draw distinction between open access publishing models and open source teaching materials
%                     \item resource challenges
%                     \item challenges of curation
%%%%%%%%%%%%%%%%%%%%%%%%%%%%%%%%


\end{document}