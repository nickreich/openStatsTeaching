%	Use the standard preamble for Beamer slides of all 
%		statsTeachR modules
%		(\input, not \include, as \include can't access 
%		things in higher-level directories since it needs 
%		write permission there, which it doesn't have; and 
%		in some settings the preamble may be in a higher-level
%		directory than the source file.)
%	This path assumes the preamble is in the parent directory,
%		modify this if that changes.
%	************************************************
%	**	LaTeX preamble to be used with all 
%	**	statsTeachR labs/handouts.
%
%	Author: Eric A. Cohen
%	Last modified: 22 August 2013
%	************************************************

\documentclass[table]{beamer}

%	Set theme (a nice plain one)
\usetheme{Malmoe}

%	Use named colors, set main color of theme
%		to match Web site color:
\definecolor{MainColor}{RGB}{10, 74, 109}
\colorlet{MainColorMedium}{MainColor!50}
\colorlet{MainColorLight}{MainColor!20}
\usecolortheme[named=MainColor]{structure} 

%	For tables
%[dvipsnames] [table]
\usepackage{xcolor}
\usepackage{tabu}	% Even fancier than tabulary
\usepackage{multirow}

%	Just for the degree symbol
\usepackage{textcomp}

%	Get rid of footline (page, author, etc. on each slide)
\setbeamertemplate{footline}{}
%	Get rid of navigation buttons
\setbeamertemplate{navigation symbols}{}

%	Make footnotes not ugly
\usepackage{hanging}
\setbeamertemplate{footnote}{\raggedright\hangpara{1em}{1}\makebox[1em][l]{\insertfootnotemark}\footnotesize\insertfootnotetext\par}

%	Text style for code snippets inline in text:
\newcommand{\codeInline}[1]{\texttt{#1}}

%	Text style for emphasis stronger than \emph:
%		(Note, this doesn't toggle the way \emph does.
%			(Note, this can be done, didn't seem worth the trouble.))
\newcommand{\strong}[1]{{\bfseries{#1}}}


%	******	Define title page	**********************
\setbeamertemplate{title page}{
	{\color{MainColor}
	% There must be a better way than this -vspace at
	%	 the top and bottom of the page to reduce the 
	%	 bottom margin, but I can't find one that works.
	\vspace{-6em}

	% Go to a lot of trouble to get the title in a
	%	nice box, since customizing a beamer block
	%	does not entirely work here (I don't know why)
	\newlength{\titleBoxWidth}
	\setlength{\titleBoxWidth}{\textwidth}
	\addtolength{\titleBoxWidth}{-2.0em}
	\setlength{\fboxsep}{1.0em}
	\setlength{\fboxrule}{0pt}
	\fcolorbox{MainColor!25}{MainColor!25}{
		\parbox{\titleBoxWidth}{
			\raggedright
			\LARGE\textbf{\inserttitle}
		}	% end parbox
	}	% end fcolorbox

	\vfill
	\small{Author: \insertauthor}
	\vspace{\baselineskip}

	\small{\Course}

	\small{\Instructor}
	\vspace{\baselineskip}

	%\small{\emph{This material is part of the \strong{statsTeachR} project}}

	\vspace{0.33\baselineskip}\scriptsize{\emph{\LicenseText}}





		\vspace{-15em}

	}	% end color
	\clearpage
}	% end define title page




%	The following variables are assumed by the standard preamble:
%	Global variable containing module name:
\title{Open Resources for Teaching Statistics }
%	Global variable containing module shortname:
%		(Currently unused, may be used in future.)
\newcommand{\ModuleShortname}{}
%	Global variable containing author name:
\author{Andrew Bray, Nicholas G Reich}
%	Global variable containing text of license terms:
\newcommand{\LicenseText}{Made available under the Creative Commons Attribution-ShareAlike 3.0 Unported License: http://creativecommons.org/licenses/by-sa/3.0/deed.en\textunderscore US }
%	Instructor: optional, can leave blank.
%		Recommended format: {Instructor: Jane Doe}
\newcommand{\Instructor}{UMass-Amherst Prbability and Statistics Seminar Series}
%	Course: optional, can leave blank.
%		Recommended format: {Course: Biostatistics 101}
\newcommand{\Course}{7 April 2014}



%	******	Document body begins here	**********************

\begin{document}

%	Title page
\begin{frame}[plain]
	\titlepage
\end{frame}

%	******	Everything through the above line must be placed at
%		the top of any TeX file using the statsTeachR standard
%		beamer preamble. 



%%%%%%%%%%%%%%%%%%%%%%%%%%%%%%%%

\begin{frame}{Overview}

	\begin{block}{Header here?}

		\begin{itemize}

			\item draw distinction between open access publishing models and open source teaching materials
                        \item resource challenges
                        \item challenges of curation
		\end{itemize}

	\end{block}

\end{frame}

%%%%%%%%%%%%%%%%%%%%%%%%%%%%%%%%


\begin{frame}{OpenIntro}

About 10 minutes of content

\end{frame}

%%%%%%%%%%%%%%%%%%%%%%%%%%%%%%%%


\begin{frame}{What is statsTeachR.org?}

\begin{block}{The basics}
\begin{itemize}
        \item a new, open-access, online repository with modular lesson plans
        \item materials targeted at undergraduates and graduates 
        \item topics: stats, biostats, statistical computing with R
\end{itemize}
\end{block}

\begin{block}{Modules}
\begin{itemize}
        \item each curricular ``module'' focuses on teaching a particular statistical subject or concept
        \item can be browsed \`a la carte
\end{itemize}
\end{block}

\end{frame}

%%%%%%%%%%%%%%%%%%%%%%%%%%%%%%%%

\begin{frame}{statsTeachR.org}

Website tour: \href{http://statsTeachR.org}{statsTeachR.org}

\end{frame}

%%%%%%%%%%%%%%%%%%%%%%%%%%%%%%%%

\begin{frame}{statsTeachR enables collaborative curriculum development}

\begin{block}{Some ongoing experiments for building curriculum}
\begin{itemize}
        \item graduate students are creating modules as final projects in core Biostat classes at UMass
        \item co-development of materials for similar classes (UMass-Amherst and Columbia Univ) 
\end{itemize}
\end{block}


\end{frame}

%%%%%%%%%%%%%%%%%%%%%%%%%%%%%%%%

\begin{frame}{Discussion questions}

\begin{itemize}
        \item where does the time/money/person-hours come from to generate these resources?
        \item how to incentivize these kinds of contributions appropriately?
        \item how to control quality of resources?
\end{itemize}
\end{frame}
%%%%%%%%%%%%%%%%%%%%%%%%%%%%%%%%


\end{document}